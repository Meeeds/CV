%% start of file `template.tex'.
%% Copyright 2006-2015 Xavier Danaux (xdanaux@gmail.com), 2020-2021 moderncv maintainers (github.com/moderncv).
%
% This work may be distributed and/or modified under the
% conditions of the LaTeX Project Public License version 1.3c,
% available at http://www.latex-project.org/lppl/.


\documentclass[11pt,a4paper,sans]{moderncv}        % possible options include font size ('10pt', '11pt' and '12pt'), paper size ('a4paper', 'letterpaper', 'a5paper', 'legalpaper', 'executivepaper' and 'landscape') and font family ('sans' and 'roman')

% moderncv themes
\moderncvstyle{classic}                             % style options are 'casual' (default), 'classic', 'banking', 'oldstyle' and 'fancy'
\moderncvcolor{blue}                               % color options 'black', 'blue' (default), 'burgundy', 'green', 'grey', 'orange', 'purple' and 'red'
%\renewcommand{\familydefault}{\sfdefault}         % to set the default font; use '\sfdefault' for the default sans serif font, '\rmdefault' for the default roman one, or any tex font name
%\nopagenumbers{}                                  % uncomment to suppress automatic page numbering for CVs longer than one page

% character encoding
%\usepackage[utf8]{inputenc}                       % if you are not using xelatex ou lualatex, replace by the encoding you are using
%\usepackage{CJKutf8}                              % if you need to use CJK to typeset your resume in Chinese, Japanese or Korean

% adjust the page margins
\usepackage[scale=0.75]{geometry}
\setlength{\footskip}{136.00005pt}                 % depending on the amount of information in the footer, you need to change this value. comment this line out and set it to the size given in the warning
%\setlength{\hintscolumnwidth}{3cm}                % if you want to change the width of the column with the dates
%\setlength{\makecvheadnamewidth}{10cm}            % for the 'classic' style, if you want to force the width allocated to your name and avoid line breaks. be careful though, the length is normally calculated to avoid any overlap with your personal info; use this at your own typographical risks...

% font loading
% for luatex and xetex, do not use inputenc and fontenc
% see https://tex.stackexchange.com/a/496643
\ifxetexorluatex
  \usepackage{fontspec}
  \usepackage{unicode-math}
  \defaultfontfeatures{Ligatures=TeX}
  \setmainfont{Latin Modern Roman}
  \setsansfont{Latin Modern Sans}
  \setmonofont{Latin Modern Mono}
  \setmathfont{Latin Modern Math}
\else
  \usepackage[utf8]{inputenc}
  \usepackage[T1]{fontenc}
  \usepackage{lmodern}
\fi

% personal data
\firstname{Louis}
\familyname{Barjon}
% \title{Resumé title}                          % optional, remove / comment the line if not wanted
\title{\small 13 octobre 1987}
\address{965 chemin du puy}{06600 Antibes}    % optional, remove / comment the line if not wanted
\mobile{06~86~67~89~06}                     % optional, remove / comment the line if not wanted
% \phone{+2~(345)~678~901}                     % optional, remove / comment the line if not wanted
% \fax{+3~(456)~789~012}                        % optional, remove / comment the line if not wanted
\email{louis.barjon@gmail.com}                  % optional, remove / comment the line if not wanted
\photo[64pt][0.4pt]{picture}                       % optional, remove / comment the line if not wanted; '64pt' is the height the picture must be resized to, 0.4pt is the thickness of the frame around it (put it to 0pt for no frame) and 'picture' is the name of the picture file

% bibliography adjustments (only useful if you make citations in your resume, or print a list of publications using BibTeX)
%   to show numerical labels in the bibliography (default is to show no labels)
%\makeatletter\renewcommand*{\bibliographyitemlabel}{\@biblabel{\arabic{enumiv}}}\makeatother
\renewcommand*{\bibliographyitemlabel}{[\arabic{enumiv}]}
%   to redefine the bibliography heading string ("Publications")
%\renewcommand{\refname}{Articles}

% bibliography with mutiple entries
%\usepackage{multibib}
%\newcites{book,misc}{{Books},{Others}}
%----------------------------------------------------------------------------------
%            content
%----------------------------------------------------------------------------------
\begin{document}

%-----       resume       ---------------------------------------------------------
\makecvtitle
%\renewcommand{\listitemsymbol}{$\circ$ }
\section{Expérience professionnelle}
%\subsection{Amadeus}
\cventry{2010 - 2022}{Amadeus}{Ingénieur développement}{}{}
{
test
}
\cvlistitem{migration}
\cvlistitem{second}
\cventry{2010}{Ubisoft}{}{}{Développeur middleware}{Développement d'un moteur physique en C++}

\section{Formation}
\cventry{2010}{Ingénieur informatique}{Ensimag}{}{}{Spécialité imagerie et réalité virtuelle}

\section{Activité d'enseignement}
\cvitem{2014}{Résistance des matériaux : Travaux pratiques et travaux dirigés à l'IUT de Lyon (\textasciitilde70h)}
\cvitem{2015 - 2017}{Informatique : Travaux pratiques sur les réseaux UNIX (niveau M2) et travaux dirigés de programmation en C (niveau L2 et L3) à l'Université de Lyon (\textasciitilde120h)}

% \section{"Awards"}
% \cvitem{}{Concours ENS}
% \cvitem{}{Bourse Normalien}

\section{Événements suivis}
\cvitem{Sep. 2014}{École d'été intitulé "Sub-Riemannian manifolds: from geodesics to hypoelliptic diffusion", CIRM, Marseille (\href{http://www.cmap.polytechnique.fr/subriemannian/cirm/}{lien}).}
\cvitem{Sep–Déc 2014}{Trimestre intitulé "Geometry, Analysis and Dynamics on sub-Riemannian Manifolds", Institut Henri Poincaré, Paris (\href{http://www.cmap.polytechnique.fr/subriemannian/index.html}{lien}).}


\section{Langues}
\cvitem{Français}{Langue maternelle}
\cvitem{Anglais}{Bonne expression orale et écrite (TOEIC niveau 920 sur 990)}

\section{Compétences informatiques}
% \cvdoubleitem{category 1}{XXX, YYY, ZZZ}{category 4}{XXX, YYY, ZZZ}
\cvitem{Programmation}{C, C++ et Latex}
\cvitem{Scientifique}{Matlab et Mathematica}
\cvitem{Autre}{Maîtrise de l'environnement UNIX}

\section{Centres d’intérêts}
\cvitem{Jeux}{Jeux de plateau, jeu de go, échecs}
\cvitem{Sciences}{Physique, mathématiques }

\cvitem{Autres}{Randonnées, énigmes }

\end{document}

