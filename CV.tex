%% start of file `template.tex'.
%% Copyright 2006-2015 Xavier Danaux (xdanaux@gmail.com), 2020-2021 moderncv maintainers (github.com/moderncv).
%
% This work may be distributed and/or modified under the
% conditions of the LaTeX Project Public License version 1.3c,
% available at http://www.latex-project.org/lppl/.


\documentclass[11pt,a4paper,sans]{moderncv}        % possible options include font size ('10pt', '11pt' and '12pt'), paper size ('a4paper', 'letterpaper', 'a5paper', 'legalpaper', 'executivepaper' and 'landscape') and font family ('sans' and 'roman')

% moderncv themes
\moderncvstyle{classic}                             % style options are 'casual' (default), 'classic', 'banking', 'oldstyle' and 'fancy'
\moderncvcolor{blue}                               % color options 'black', 'blue' (default), 'burgundy', 'green', 'grey', 'orange', 'purple' and 'red'
%\renewcommand{\familydefault}{\sfdefault}         % to set the default font; use '\sfdefault' for the default sans serif font, '\rmdefault' for the default roman one, or any tex font name
\nopagenumbers{}                                  % uncomment to suppress automatic page numbering for CVs longer than one page

% character encoding
%\usepackage[utf8]{inputenc}                       % if you are not using xelatex ou lualatex, replace by the encoding you are using
%\usepackage{CJKutf8}                              % if you need to use CJK to typeset your resume in Chinese, Japanese or Korean

% adjust the page margins
\usepackage[top=1cm,bottom=1cm,left=2cm,right=2cm]{geometry}
\recomputelengths
%\setlength{\footskip}{12.00005pt}                 % depending on the amount of information in the footer, you need to change this value. comment this line out and set it to the size given in the warning
%\setlength{\hintscolumnwidth}{3cm}                % if you want to change the width of the column with the dates
%\setlength{\makecvheadnamewidth}{15cm}            % for the 'classic' style, if you want to force the width allocated to your name and avoid line breaks. be careful though, the length is normally calculated to avoid any overlap with your personal info; use this at your own typographical risks...

% font loading
% for luatex and xetex, do not use inputenc and fontenc
% see https://tex.stackexchange.com/a/496643
\ifxetexorluatex
  \usepackage{fontspec}
  \usepackage{unicode-math}
  \defaultfontfeatures{Ligatures=TeX}
  \setmainfont{Latin Modern Roman}
  \setsansfont{Latin Modern Sans}
  \setmonofont{Latin Modern Mono}
  \setmathfont{Latin Modern Math}
\else
  \usepackage[utf8]{inputenc}
  \usepackage[T1]{fontenc}
  \usepackage{lmodern}
\fi

% personal data
\firstname{Louis}
\familyname{Barjon}
%\title{Resu title}                          % optional, remove / comment the line if not wanted
\title{\small 13 octobre 1987}
% \address{965 chemin du puy}{06600 Antibes}    % optional, remove / comment the line if not wanted
\social[linkedin]{louis-barjon}
\mobile{06~86~67~89~06}                     % optional, remove / comment the line if not wanted
% \phone{+2~(345)~678~901}                     % optional, remove / comment the line if not wanted
% \fax{+3~(456)~789~012}                        % optional, remove / comment the line if not wanted
\email{louis.barjon@gmail.com}                  % optional, remove / comment the line if not wanted
\photo[64pt][0.4pt]{picture}                       % optional, remove / comment the line if not wanted; '64pt' is the height the picture must be resized to, 0.4pt is the thickness of the frame around it (put it to 0pt for no frame) and 'picture' is the name of the picture file
\quote{Senior développeur cloud \& C\texttt{++}}
% bibliography adjustments (only useful if you make citations in your resume, or print a list of publications using BibTeX)
%   to show numerical labels in the bibliography (default is to show no labels)
%\makeatletter\renewcommand*{\bibliographyitemlabel}{\@biblabel{\arabic{enumiv}}}\makeatother
%\renewcommand*{\bibliographyitemlabel}{[\arabic{enumiv}]}
%   to redefine the bibliography heading string ("Publications")
%\renewcommand{\refname}{Articles}

% bibliography with mutiple entries
%\usepackage{multibib}
%\newcites{book,misc}{{Books},{Others}}
\usepackage{graphicx,calc}
\newlength\myheight
\newlength\mydepth
\settototalheight\myheight{Xygp}
\settodepth\mydepth{Xygp}
\setlength\fboxsep{0pt}
\newcommand*\inlinegraphics[1]{%
  \settototalheight\myheight{Xygp}%
  \settodepth\mydepth{Xygp}%
  \raisebox{-\mydepth}{\includegraphics[height=\myheight]{#1}}%
}

%----------------------------------------------------------------------------------
%            content
%----------------------------------------------------------------------------------
\begin{document}

%-----       resume       ---------------------------------------------------------
\makecvtitle
%\renewcommand{\listitemsymbol}{$\circ$ }
\section{Expérience professionnelle}
\subsection{\textbf{Amadeus}, Antibes}
\cventry{2020 - }{Sénior développeur cloud}{}{}{}
{}
\cvlistitem{Migration du produit paiement au private cloud et Microsoft Azure}
\cvlistitem{Mise en place de code quality avec SonarQube 
%\inlinegraphics{sonar}
}
\cventry{2016 - 2020}{Développeur cloud}{}{}{}
{}
\cvlistitem{Migration du produit Flight Search à Google Cloud Engine à l'aide d'OpenShift / Kubernetes / Docker. Migration des données sur couchbase avec le client C\texttt{++}}
\cvlistitem{Unit test avec 	Gmok/Gtest}
\cventry{2013 - 2016}{Développeur C\texttt{++}}{}{}{}
{}
\cvlistitem{Développement C\texttt{++} du produit Flight Search avec de fortes contraintes de performances (Plus de 100k tps, réponse $\sim$100ms)}
\cvlistitem{Optimisation de l'algo MCT (Minimum Connecting Time) et division par 5 du CPU}

%\cvlistitem{migration}
%\cvlistitem{second}
\cventry{2010 - 2013}{Développeur C\texttt{++}}{débutant}{}{}
{
Migration technique des données vols depuis le système TPF (IBM) vers une architecture micro service développée en interne.
}
\subsection{\textbf{Ubisoft}, Montpellier}
\cventry{2010}{Développeur middleware}{}{}{}{Développement d'un moteur physique en C\texttt{++}}

\section{Formation}
\cventry{2010}{Ingénieur informatique}{Ensimag}{}{}{Spécialité: imagerie et réalité virtuelle}

\section{Compétences}
%% \cvitem{Skill matrix}{Alternatively, provide a skill matrix to show off your skills}
%% Skill matrix as an alternative to rate one's skills, computer or other. 

%% Adjusts width of skill matrix columns. 
%% Usage \setcvskillcolumns[<width>][<factor>][<exp_width>]
%% <width>, <exp_width> should be lengths smaller than \textwidth, <factor> needs to be between 0 and 1.
%% Examples:
% \setcvskillcolumns[5em][][]%    adjust first column. Same as \setcvskillcolumns[5em]
% \setcvskillcolumns[][0.45][]%   adjust third (skill) column. Same as \setcvskillcolumns[][0.45]
%\setcvskillcolumns[][0.1][\widthof{``Year''}]%     adjust fourth (years) column.
% \setcvskillcolumns[][0.45][\widthof{``Year''}]%
% \setcvskillcolumns[\widthof{``Languag''}][0.48][]
% \setcvskillcolumns[\widthof{``Languag''}]%

\setcvskillcolumns[][][1pt]%     adjust fourth (years) column.

%% Adjusts width of legend columns. Usage \setcvskilllegendcolumns[<width>][<factor>]
%% <factor> needs to be between 0 and 1. <width> should be a length smaller than \textwidth
%% Examples:
% \setcvskilllegendcolumns[][0.45]
% \setcvskilllegendcolumns[\widthof{``Legend''}][0.45]
% \setcvskilllegendcolumns[0ex][0.46]% this is usefull for the banking style

%% Add a legend if you are using \cvskill{<1-5>} command or \cvskillentry
%% Usage \cvskilllegend[*][<post_padding>][<first_level>][<second_level>][<third_level>][<fourth_level>][<fifth_level>]{<name>}
% \cvskilllegend % insert default legend without lines
%% \cvskilllegend*[1em]{}% adjust post spacing
% \cvskilllegend*{Legend}%  Alternatively add a description string
%% adjust the legend entries for other languages, here German
% \cvskilllegend[0.2em][Grundkenntnisse][Grundkenntnisse und eigene Erfahrung in Projekten][Umfangreiche Erfahrung in Projekten][Vertiefte Expertenkenntnisse][Experte\,/\,Spezialist]{Legende}

%% Alternative legend style with the first three skill levels in one column
%% Usage \cvskillplainlegend[*][<post_padding>][<first_level>][<second_level>][<third_level>][<fourth_level>][<fifth_level>]{<name>}
% \setcvskilllegendcolumns[][0.6]%  works for classic, casual, banking
% \setcvskilllegendcolumns[][0.55]%  works better for oldstyle and fancy
% \cvskillplainlegend{}
% \cvskillplainlegend[0.2em][Grundkenntnisse][Grundkenntnisse und eigene Erfahrung in Projekten][Umfangreiche Erfahrung in Projekten][Vertiefte Expertenkenntnisse][Experte/Guru]{Legende}

%% Add a head of the skill matrix table with descriptions.
%% Usage \cvskillhead[<post_padding>][<Level>][<Skill>][<Years>][<Comment>]%
\cvskillhead[-0.1em][Niveau][][][Commentaire]%   this inserts the standard legend in english and adjust padding
%% Adjust head of the skill matrix for other languages
% \cvskillhead[0.25em][Level][F\"ahigkeit][Jahre][Bemerkung]

%% \cvskillentry[*][<post_padding>]{<skill_cathegory>}{<0-5>}{<skill_name>}{<years_of_experience>}{<comment>}% 
%% Example usages:
\cvskillentry*{Langues:}{3}{Anglais}{}{Used every day for work, social, and science}
\cvskillentry*{Languages:}{4}{C\texttt{++}}{}{Dev backend, optim. de performances}
\cvskillentry{}{2}{python, go, bash}{}{Développement d'outils}
\cvskillentry{}{1}{sql, couchbase}{}{Gestion et transfert des données}
\cvskillentry*{Cloud:}{3}{openshift, helm, kube, docker}{}{Déploiement de soft sur gce/Azure}
%\cvskillentry*{OS:}{4}{Linux, Windows}{}{Dev sur server linux et sur VScode}
\cvskillentry*{Méthodes:}{3}{Scrum, Kanban, Safe, Jira}{}{Travail en agile depuis 2015}
% \cvitem{\textbackslash{cvskill}:}{Skills can be visually expressed by the \textbackslash{cvskill} command, e.g. \cvskill{2}}
\vspace{15pt}
\cvitem{Autres:}{mongo, kafka, perl, java, javascript, git, VScode, Eclipse, \LaTeX}


\section{Centres d’intérêts}
\cvitem{Jeux}{Jeux de plateau, jeu de go, échecs}
\cvitem{Sciences}{Physique, mathématiques }
\cvitem{Autres}{Randonnées, énigmes }

\end{document}

